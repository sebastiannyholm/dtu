\section{Opgave 1.2 - NextPrime}

Idéen bag programmet er en simpel generering af primtalsrækken med den enkelte forskel, at der ikke køres igennem alle positive integers. Den givne \textit{integer} (heltal) $n$ udgør således begyndelsesværdien for primtalsgenereringen, hvorefter det først fundne primtal printes til konsollen som svar. Primtalsalgoritmen brugt, I dette tilfælde, går ud på først at: 

\begin{itemize}
	\item Ignorere alle multiplum af 2 - Der findes ingen lige primtal støre end 2.
	\item Hvis $n<2$, så er det første primtal efter $n$ lig 2. 
	\begin{itemize}
		\item Primtal er kun defineret for \textbf{positive} heltal.
	\end{itemize}
	\item Kun heltallet $n=2$ vil kunne resultere i udskrivningen af primtallet 3.
\end{itemize}

\noindent Den mere bærende del af algoritmen, og dermed programmet, består i at teste for divisorere $d<\sqrt{n}$. Grundet den måde faktorere af $n$ gentages efter $\sqrt{n}*\sqrt{n}$, er der ingen grund til at iterere yderligere end til dette punkt ($d<\sqrt{n}$). Dette kommer af at multiplikation er \emph{kommutativt}:

\begin{equation*}
	36 = 18*2 = 12*3 9*4 = \sqrt{36}*\sqrt{36} = 4*9 = 3*12 = 2*18 = 36
\end{equation*}

\noindent Vi tester altså $n$ for alle \textit{ulige} divisorere $d<\sqrt{n}$, hvis ingen findes er $n$ et primtal.\footnote{Se bilag - Kildekode: NextPrime.java}