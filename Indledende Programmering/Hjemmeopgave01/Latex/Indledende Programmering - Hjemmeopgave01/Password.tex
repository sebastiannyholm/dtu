\section{Opgave 1.1 - Password}

Password class'en er lavet med en enkelt public metode, så den kan blive brugt udefra. Metoden er lavet, så den returner en boolean, altså false eller true, da metoden kun skal afgøre om det indtastede password overholder de givne regler.

Vi har lavet metoden, så den gennemgår alle reglerne en efter en og hvis den støder ind i en fejl undervejs, returner den false med det samme, så man undgår at gennemgå resten af koden, da passwordet allerede har fejlet.

Alle reglerne bliver tjekket af if statements, men de fleste steder skal alle tegn i passwordet tjekkes, hvilket bliver gjort ved hjælp af en for lykke.

\begin{center}
    for (int i = 0; i \textless passLength; i++)
\end{center}

I for lykken er der if statements inden i, som afgør om passwordet holder og hvis det fejler eller holder, alt efter hvad reglen er, breaker lykken, så man ikke behøver køre den hele vejen igennem.